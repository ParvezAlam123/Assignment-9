\documentclass{article}
\usepackage[utf8]{inputenc}
\usepackage{multicol}
\usepackage{hyperref}
\usepackage{amsmath}
\usepackage{mathtools}
\usepackage{graphicx}
\graphicspath{{./}}
\setlength{\columnsep}{2em}
\usepackage{geometry}
\geometry{
 a4paper,
 left=5mm,
 right=5mm,
 top=5mm,
 }

\title{Assignment 9}
\author{Parvez Alam : AI21RESCH01005 }
\date{March 2021}

\begin{document}

\maketitle
\begin{multicols}{2}
\begin{center}
    \begin{tabular}{|p{5cm}|}
    \hline
        Python code link:\url{https://github.com/ParvezAlam123/Assignment-9} \\
        \hline
    \end{tabular}
\end{center}


\section{Prob. Misc. 5.29}
\textbf{Let a pair of dice be thrown and the random variable X be the sum of the numbers that appear on the two dice. Find the mean or expectation of X.} \\ \\
\textbf{Solution:} 
Let \(X_1\) be random variable for first dice and\( X_2\) be random variable for second dice
\begin{align}
    X_1, X_2 &\in \{1,2,3,4,5,6\}  \nonumber \\
    X &= X_1+X_2 \nonumber  \nonumber \\
    X &\in \{1,2,3,4,5,6,7,8,9,10,11,12\} \nonumber  \\
    p_{X_i}(n)&=\begin{dcases}
           \frac{1}{6} & 1 \leq X_i \leq 6 \\
           0 & otherwise
           \end{dcases} \nonumber \\
    p_{X}(n)&=\Pr(X_1+X_2=n) \nonumber \\
            &=\Pr(X_1=n-X_2) \nonumber \\
            &=\sum_k \Pr(X_1=n-k|X_2=k)p_{X_2}(k) \nonumber \\
            &=\sum_k p_{X_1}(n-k)p_{X_2}(k) \nonumber \\
            &=\frac{1}{6}\sum_{k=1}^6 p_{X_1}(n-k) \nonumber \\
            &=\frac{1}{6}\sum_{k=n-6}^{n-1}p_{X_1}(k) \nonumber 
\end{align}
\[p_{X}(n)=\begin{dcases}
           0 & n \leq 1 \\
           \frac{1}{6}\sum_{k=1}^{n-1}p_{X_1}(k)=\frac{n-1}{36} & 2 \leq n \leq 7 \\
           \frac{1}{6}\sum_{k=n-6}^6 p_{X_1}(k)=\frac{13-n}{36} & 7 < n \leq 12 \\
           0 & n>12
           \end{dcases}\]
\begin{align}
    E[X]&=\sum_{n=1}^{12} n p_{X}(n) \nonumber \\
        &=\sum_{n=2}^7 n\times \frac{n-1}{36}+\sum_{n=8}^{12} n \times \frac{13-n}{36} \nonumber \\
        &=\frac{2}{36}+\sum_{n=3}^7 n\times \frac{n-1}{36}+\sum_{n=8}^{12}n\times \frac{13-n}{36}\nonumber \\
        &=\frac{2}{36}+\frac{1}{36}\left(\sum_{n=3}^7n(n-1)+(n+5)(13-(n+5))\right) \nonumber \\
        &=\frac{2}{36}+\frac{1}{36}\left(\sum_{n=3}^7 n^2-n+(n+5)(8-n)\right) \nonumber \\
        &=\frac{2}{36}+\frac{1}{36}\left(\sum_{n=3}^7 n^2-n+8n-n^2+40-5n\right) \nonumber \\
        &=\frac{2}{36}+\frac{1}{36}\left(\sum_{n=3}^7 2n+40 \right) \nonumber \\
        &=\frac{1}{18}+\frac{1}{18}\left(\sum_{n=3}^7 n+20\right) \nonumber \\
        &=\frac{1}{18}\left(1+\sum_{n=3}^7 n+20 \right) \nonumber \\
        &=\frac{1}{18}\left(1+23+24+25+26+27\right) \nonumber \\
        &=7 \nonumber
\end{align}
\includegraphics[width=\columnwidth]{plot_9_2.png}
\includegraphics[width=\columnwidth]{Plot_9_3.png}
\includegraphics[width=\columnwidth]{Plot_9.png}



\end{multicols}
\end{document}
