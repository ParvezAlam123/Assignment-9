\documentclass{article}
\usepackage[utf8]{inputenc}
\usepackage{multicol}
\usepackage{hyperref}
\usepackage{amsmath}
\usepackage{graphicx}
\graphicspath{{./}}
\setlength{\columnsep}{2em}
\usepackage{geometry}
\geometry{
 a4paper,
 left=5mm,
 right=5mm,
 top=5mm,
 }

\title{Assignment 9}
\author{Parvez Alam : AI21RESCH01005 }
\date{March 2021}

\begin{document}

\maketitle
\begin{multicols}{2}
\begin{center}
    \begin{tabular}{|p{5cm}|}
    \hline
        Python code link:\url{https://github.com/ParvezAlam123/Assignment-9} \\
        \hline
    \end{tabular}
\end{center}


\section{Prob. Misc. 5.29}
\textbf{Let a pair of dice be thrown and the random variable X be the sum of the numbers that appear on the two dice. Find the mean or expectation of X.} \\ \\
\textbf{Solution:} 
Let \(X_1\) be random variable for first dice and\( X_2\) be random variable for second dice
\begin{align}
    X_1, X_2 &\in \{1,2,3,4,5,6\}  \nonumber \\
    X &= X_1+X_2 \nonumber  \nonumber \\
    X &\in \{2,3,4,5,6,7,8,9,10,11,12\} \nonumber 
\end{align}
\begin{center}
    \begin{tabular}{|p{1cm}|{4cm}|}
    \hline
        X & outcomes \\
        \hline
        2 & (1,1) \\
        \hline
        3 & (1,2),(2,1) \\
        \hline
        4 & (1,3),(2,2),(3,1) \\
        \hline
        5 & (1,4),(2,3),(3,2),(4,1) \\
        \hline
        6 & (1,5),(2,4),(3,3),(4,2),(5,1) \\
        \hline
        7 & (1,6),(2,5),(3,4),(4,3),(5,2),(6,1) \\
        \hline
        8 & (2,6),(3,5),(4,4),(5,3),(6,2) \\
        \hline 
        9 & (3,6),(4,5),(5,4),(6,3) \\
        \hline
        10 & (4,6),(5,5),(6,4) \\
        \hline
        11 & (5,6),(6,5) \\
        \hline
        12 & (6,6)  \\
        \hline
    \end{tabular}
\end{center}
\begin{align}
    P(X=2)&=P(X_1+X_2=2) \nonumber \\
          &=\frac{1}{36} \nonumber \\
    P(X=3)&=P(X_1+X_2=3) \nonumber \\
          &=\frac{2}{36} \nonumber \\
    P(X=4) &=P(X_1+X_2=4) \nonumber \\
           &=\frac{3}{36} \nonumber \\
    P(X=5) &= P(X_1+X_2) \nonumber \\
           &=\frac{4}{36} \nonumber \\
    P(X=6) &=P(X_1+X_2=6) \nonumber \\
           &=\frac{5}{36} \nonumber \\
    P(X=7) &=P(X_1+X_2=7) \nonumber \\
           &=\frac{6}{36} \nonumber \\
    P(X=8) &=P(X_1+X_2=8) \nonumber \\
           &=\frac{5}{36} \nonumber \\
    P(X=9) &=P(X_1+X_2=9) \nonumber \\
           &=\frac{4}{36} \nonumber \\
    P(X=10)&=P(X_1+X_2=10) \nonumber \\
           &=\frac{3}{36} \nonumber \\
    P(X=11) &=P(X_1+X_2=11) \nonumber \\
           &=\frac{2}{36}  \nonumber \\
    P(X=12) &=P(X_1+X_2=12) \nonumber \\
           &=\frac{1}{36} \nonumber 
\end{align}
\textbf{Expectation:}
\begin{align}
    E[X]&=\sum_{i=1}^{12}x_iP(X=x_i) \nonumber \\
        &=2\times \frac{1}{36}+3\times \frac{2}{36} \nonumber \\
        &+4\times \frac{3}{36}+5\times \frac{4}{36} \nonumber \\
        &+6\times \frac{5}{36}+7\times \frac{6}{36} \nonumber \\
        &+8\times \frac{5}{36}+9\times \frac{4}{36} \nonumber \\
        &+10\times \frac{3}{36}+11\times \frac{2}{36} \nonumber \\
        &+12\times \frac{1}{36} \nonumber \\
        &=\frac{1}{36}(2+6+12+20+30 \nonumber \\
        &+42+40+36+30+22+12)    \nonumber \\
        &=6.444 \nonumber
\end{align}
\includegraphics[width=\columnwidth]{Plot_9.png}



\end{multicols}
\end{document}
